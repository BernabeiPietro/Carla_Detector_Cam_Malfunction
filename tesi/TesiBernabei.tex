\documentclass[14pt]{extarticle}
\usepackage{graphicx}
\usepackage{hyperref}
\usepackage[english]{babel}
\usepackage[final]{pdfpages}

\usepackage[utf8x]{inputenc}
\usepackage{listings}
\usepackage{xcolor}
\usepackage{color}




\definecolor{javared}{rgb}{0.6,0,0} % for strings
\definecolor{javagreen}{rgb}{0.25,0.5,0.35} % comments
\definecolor{javapurple}{rgb}{0.5,0,0.35} % keywords
\definecolor{javadocblue}{rgb}{0.25,0.35,0.75} % javadoc
 
\lstset{language=Python,
    keywordstyle=\color{javapurple},
    basicstyle=\small,
    commentstyle=\color{javagreen},
    stringstyle=\color{javadocblue},
    showstringspaces=false,
    breaklines=true,
    frameround=ffff,
    frame=single,
    rulecolor=\color{black}
} 


\begin{document}
\title{\includegraphics{download.jpeg} \vspace{2cm} \textbf{\\Tesi di Laurea}}

\author{\texttt{Pietro Bernabei} - \texttt{matricola:6291312}\\ \texttt{Anno Accademico 2019/20}}
\date{}
\maketitle

\newpage
\section{Introduzione}
\subsection{Motivazioni}
L'avanzamento tecnologico vissuto nell'ultimo periodo,  ha portato ha un cambiamento nelle nostre vite, con l'introduzione di nuovi sistemi, applicazioni e infrastrutture, hardware e software, sulle quali si fa sempre più affidamento.
Data questa particolare dipendenza dalla loro presenza, si richiede che questi siano continuativa, libera da possibili errori o malfunzionamenti. Questi sono definiti Sistemi critici.
Un particolare sistemi critico, che verrà trattato da questa tesi, è il sistema di guida autonoma di una macchina.
Tecnologia in sviluppo negli ultimi anni prevede l'uso di sensori, per interagire e sentire il mondo esterno. Uno dei più importanti e comuni sensori usati all'interno dei sistemi a guida autonoma, sono le telecamere." Dispositivo elettronico in grado di acquisire immagini bidimensionali in sequenza a velocità di cattura prefissate, solitamente nella gamma visibile dello spettro elettromagnetico"["wikipedia-telecamere"]. Questo può subire diversi malfunzionamenti, come la sfuocatura, il congelamente del vetro, la rottura dei pixels, e altri ancora, i quali portano a malfunzionamenti nel momento decisionale.
\subsection{Obbiettivo}
Detto ciò, l'obbiettivo della seguente tesi è di andare a definire un sistema software, detto detector, in grado di andare a rilevare, nel flusso di immagini rilevate in input dalla telecamara del sistema, i malfunzionamenti nel sistema di acquisizione. 
\subsection{Organizzazione del lavoro}
Il detector sviluppato nella seguente tesi, si basa sull'implementazione di una intelligenza artificiale, CNN (convulutional Neureal Network), addestrata con un dataset di immagini acquisite in  precedenza, per poi essere inserito all'interno del simulatore
\section{Fondamenti}


\section{Costruzione del dataset}

\section{Costruzione del detector}

\section{Esecuzione e risultati}

\section{Conclusioni e lavori futuri}

\section{A Manuale utente}

\end{document}